\documentclass{article}
\usepackage[top=.8in, bottom=.8in, left=.8in, right=.8in]{geometry}
\usepackage{fancyhdr}
\usepackage{multicol}
\usepackage[fleqn]{mathtools}
\DeclarePairedDelimiter\abs{\lvert}{\rvert}

\fancyhead[L]{Cheatsheet by Tiffany Jaya}
\fancyhead[R]{Fall 2017}

\pagestyle{fancy}
\usepackage{Sweave}
\begin{document}
\begin{small}
\Sconcordance{concordance:statistics-cheatsheet.tex:statistics-cheatsheet.Rnw:%
1 11 1 1 0 382 1}


% =============================================================================
% Page 1
% =============================================================================

Rules of \textbf{Probability}

\begin{multicols}{2}
% -----------------------------------------------------------------------------
% Page 1, LH Column #1
% -----------------------------------------------------------------------------
\begin{align*}
P(X \cup Y) & = P(X) + P(Y) - P(X \cap Y) \\\\
P(X \cap Y) = P(Y \cap X) & = P(X) \cdot P(Y \mid X) \\
                          & = P(Y) \cdot P(X \mid Y) \\\\
P(Y)        & = 1 - P(Y') \\
            & = P(Y \cap A) + P(Y \cap B) + ... + P(Y \cap Z)\\
            & = P(Y \cap X) + P(Y \cap \ !X) \\
            & = P(X) \cdot P(Y \mid X) + P(!X) \cdot P(Y \mid \ !X) \\
            & = P(Y) \cdot P(X \mid Y) + P(Y) \cdot P(!X \mid Y) \\\\
P(X \mid Y) & = \frac{P(X \cap Y)}{P(Y)} \\
            & = \frac{P(X) \cdot P(Y \mid X)}{P(Y)} = \frac{P(Y) \cdot P(X \mid Y)}{P(Y)} \\
            & = \frac{P(X) \cdot P(Y \mid X)}{P(X) \cdot P(Y \mid X) + P(!X) \cdot P(Y \mid \ !X)} 
              = \frac{P(Y) \cdot P(X \mid Y)}{P(X) \cdot P(Y \mid X) + P(!X) \cdot P(Y \mid \ !X)}\\
            & = \frac{P(X) \cdot P(Y \mid X)}{P(Y) \cdot P(X \mid Y) + P(Y) \cdot P(!X \mid Y)}
              = \frac{P(Y) \cdot P(X \mid Y)}{P(Y) \cdot P(X \mid Y) + P(Y) \cdot P(!X \mid Y)}\\\\
P(X \cap Y \cap Z) & = P(X) \cdot P(Y \mid X) \cdot P(Z \mid X \cap Y)\\
                   & = P(Y) \cdot P(X \mid Y) \cdot P(Z \mid X \cap Y)\\\\
P(X \cup Y \cup Z) & = P(X) + P(Y) + P(Z) - P(X \cap Y) - P(X \cap Z) - P(Y \cap Z) + P(X \cap Y \cap Z)\\\\
P(A) + P(B) + P(C) & = 1 \ if \ mutually \ exlusive
\end{align*}

\columnbreak
% -----------------------------------------------------------------------------
% Page 1, RH Column #1
% -----------------------------------------------------------------------------
\textbf{Independent}
\begin{align*}
\\\\
P(X \cap Y) = P(Y \cap X) & = P(X) \cdot P(Y) \\
            & = P(Y) \cdot P(X) \\\\
\\\\\\\\\\\\
P(X \mid Y) & = P(X) 
\end{align*}
\end{multicols}
\begin{multicols}{2}
% -----------------------------------------------------------------------------
% Page 1, LH Column #2
% -----------------------------------------------------------------------------
\textbf{discrete}
\begin{enumerate}
\item\textbf{probability mass function (PMF)}

list all possible probabilities of x

$p_X(x) = P(X = x)$

  \begin{enumerate}
  \item probability of x must be in [0, 1]
  \item sum of all probabilities must equal to 1
  \item R: hist(x) \\ with $x$ in x-axis and $p_X(x)$ in y-axis
  \end{enumerate}

\item\textbf{cumulative distribution function (CDF)}

probability that a random variable is less than or equal to x

$F_X(x) = P(X \leq x)$

  \begin{enumerate}
  \item $\lim_{x\to-\infty} F_X(x) = 0$
  \item $\lim_{x\to\infty} F_X(x) = 1$
  \item non-decreasing
  \end{enumerate}

\item\textbf{$E_X(x)$ = theoretical mean}

For random variable x, 

$E_X(x) = \sum_{all \ x} x \cdot p_X(x)$

For function g(x), 

$E_X(g(x)) = \sum_{all \ x} g(x) \cdot p_X(x)$

\end{enumerate}

\columnbreak
% -----------------------------------------------------------------------------
% Page 1, RH Column #2
% -----------------------------------------------------------------------------
\textbf{continuous}
\begin{enumerate}
\item\textbf{probability density function (PDF)}

list all possible probabilities of x

$f_X(x) = \frac{\partial}{\partial x} \ F_X(x)$

  \begin{enumerate}
  \item $f_X(x) \ge 0$ for all x
  \item area under the curve is equal to 1
  \item R: plot(density(x)) \\ with $x$ in x-axis and $f_X(x)$ in y-axis
  \end{enumerate}

\item\textbf{cumulative distribution function (CDF)}

probability that a random variable is less than or equal to x

$F_X(x) = P(X \leq x) = \int_{-\infty}^{x} f_X(x) \ dx$ 

$P(a \le X \le b) = F_X(b) - F_X(a) = \int_{a}^{b} f_X(x) \ dx$

  \begin{enumerate}
  \item $\lim_{x\to-\infty} F_X(x) = 0$
  \item $\lim_{x\to\infty} F_X(x) = 1$
  \item non-decreasing
  \end{enumerate}

\item\textbf{$E_X(x)$ = theoretical mean}

For random variable x, 

$E_X(x) = \int_{-\infty}^{\infty} x \cdot f_X(x)$

For function g(x), 

$E_X(g(x)) = \int_{-\infty}^{\infty} g(x) \cdot f_X(x)$

\end{enumerate}

\end{multicols}

\pagebreak

% =============================================================================
% Page 2
% =============================================================================

\begin{multicols}{2}
% -----------------------------------------------------------------------------
% Page 2, LH Column #1
% -----------------------------------------------------------------------------
\begin{enumerate}
\item Rules of \textbf{Expectation} 
  \begin{enumerate}
  \item $E(aX + b) = a \cdot E(X) + b$
  \item $E(Y \mid X) = \int_{-\infty}^{\infty} y \cdot f_{Y \mid X}(y \mid x) \ dy$
  \item $E(XY) = E(E(XY \mid X)) = E(X \cdot E(Y \mid X))$
  \item $E(XY) == E(X) \cdot E(Y)$ if X, Y independent
  \item $E(XY \mid X) = X \cdot E(Y \mid X)$
  \item $E(Y) = E(E(Y \mid X))$
  \item $E_X(x^2) = \int_{-\infty}^{\infty} x^2 \cdot f_X(x)$
  \end{enumerate}

\item Rules of \textbf{Variance}
  \begin{enumerate}
  \item $var(X) = \sigma^2 = E(X^2) - E(X)^2$
  \item $var(X + c) = var(X)$
  \item $var(cX) = c^2 \cdot var(X)$
  \item $var(X - Y) = var(X) + var(Y)$
  \item $var(X + Y) = var(X) + var(Y) + 2 \cdot cov(X, Y)$
  \item $var(X + Y) = var(X) + var(Y)$ if\\X, Y independent
  \end{enumerate}

\item Rules of \textbf{Standard Deviation}
  \begin{enumerate}
  \item $\sigma = \sqrt{var(X)}$
  \end{enumerate}

\item Rules of \textbf{Normal Distribution}
  \begin{enumerate}
  \item $E(X) = 0$
  \item $E(X^2) = 1$
  \item $E(X)^2 = 0$
  \item $\sigma = 1$
  \item $var(X) = 1$
  \end{enumerate}
\item Rules of \textbf{Joint Distribution}

Remember the rules of probability above\\(for discrete) applies here. 

  \begin{enumerate}
  \item\textbf{Marginal}
  \begin{align*}
  f_X(x) & = \int_{-\infty}^{\infty} f_{X,Y}(x, y) \ dy \\
         & = \int_{-\infty}^{\infty} f_X(x) \cdot f_{Y \mid X}(y \mid x) \ dy \\
  f_Y(y) & = \int_{-\infty}^{\infty} f_{X,Y}(x, y) \ dx \\
         & = \int_{-\infty}^{\infty} f_Y(y) \cdot f_{X \mid Y}(x \mid y) \ dx
  \end{align*}
  \item $f_{X \mid Y} = \frac{f_{X, Y}(X, Y)}{f_Y(y)}$
  \item $f_{X, Y} == f_X(x) \cdot f_Y(y)$ if X, Y independent
  \end{enumerate}

\item Rules of \textbf{Covariance}

Measure the strength of linear relationship b/w 2 variables

  \begin{enumerate}
  \item $cov(X, Y) = E((X - E(X))\cdot(Y - E(Y)))\\= E(XY) - E(X) \cdot E(Y)$
  \item $cov(X, Y) = 0$ if X, Y independent \\
        but cov == 0 does not mean X,Y independent
  \item $cov(X, X) = var(X)$
  \item $cov(aX, bY) = ab \cdot cov(XY)$
  \item $cov(X, Y+Z) = cov(X, Y) + cov(X, Z)$
  \end{enumerate}

\item Rules of \textbf{Correlation}

Normalize Covariance by scaling it b/w -1 and 1

  \begin{enumerate}
  \item $corr(X, Y) = \frac{cov(X, Y)}{\sigma_X \cdot \sigma_Y}$
  \end{enumerate}

\end{enumerate}

\columnbreak
% -----------------------------------------------------------------------------
% Page 2, RH Column #1
% -----------------------------------------------------------------------------
\begin{enumerate}
  \setcounter{enumi}{8}
  \item Rules of \textbf{Complement}
  \begin{enumerate}
  \item $P(X > 3) = 1 - P(X \le 3)$
  \item $P(2 \le X \le 5) = P(X \le 5) - P(X \le 1)$
  \item $P(2 < X < 5) \\= P(3 \le X \le 4) = P(X \le 4) - P(X \le 2)$
  \item $P(A \mid B') = 1 - P(A \mid B)$
  \item $P(A' \mid B') = 1 - P(A \mid B')$
  \end{enumerate}
  
  \item Rules of \textbf{Sum}
  \begin{enumerate}
  \item Finite Arithmetic \\
        i.e. 3, 7, 11, ..., 51 \\
        $S_n = \frac{n}{2}(a_1 + a_n) = n[a_1 + \frac{n-1}{2}d]$
  \item Finite Geometric, $r = \frac{a_n}{a_n - 1}$ \\
        i.e. 1, 2, 4, ..., 64 \\
        $S_n = a_1 \frac{1 - r^n}{1 - r} = \frac{a_1 - ra_n}{1 - r}$
  \item Infinite Geometric, $r = \frac{a_n}{a_n - 1}$, -1 < r < 1\\
        i.e. $3, 1, \frac{1}{3}, \frac{1}{9}, \frac{1}{27}, ...$ \\
        $S = \frac{a_1}{1 - r}$
  \item Infinite Arithmetic and Infinite Geometric w/ $r \le -1$ or $1 \le r$ \\
        No formula
  \end{enumerate}
  
  \item Rules of \textbf{Log} 
  \begin{align*}
  log_2(2^k) & = log_2(9801) \\
          k  & = \frac{log_{10}(9801)}{log_{10}(2)}
  \end{align*}
  
\end{enumerate}

\textbf{Discrete Joint Probability Table}\\
\begin{tabular}{ |c|c|c|c|c|c| }
\hline
& $y_1$ & $y_2$ & ... & $y_n$ & \\
\hline
$x_1$ & $f_{X,Y}(x_1, y_1)$ &  $f_{X,Y}(x_1, y_2)$ & ... & $f_{X,Y}(x_1, y_n)$ & $f_X(x_1)$\\
\hline
$x_2$ & $f_{X,Y}(x_2, y_1)$ &  $f_{X,Y}(x_2, y_2)$ & ... & $f_{X,Y}(x_2, y_n)$ & $f_X(x_2)$\\
\hline
...   & ...                 &  ...                 & ... & ...                 & ... \\
\hline
$x_m$ & $f_{X,Y}(x_m, y_1)$ &  $f_{X,Y}(x_m, y_2)$ & ... & $f_{X,Y}(x_m, y_n)$ & $f_X(x_m)$\\
\hline
      & $f_Y(y_1)$    &  $f_Y(y_2)$    & ... & $f_Y(y_n)$    & 1\\
\hline
\end{tabular}
.\\\\
\textbf{Variance-Covariance Matrix}\\
\begin{tabular}{ |c|c|c|c| }
\hline
$var(x_1) $ & $cov(x_1, x_2)$ & ... & $cov(x_1, x_n)$  \\
\hline
$cov(x_2, x_1)$ & $var(x_2) $ & ... & $cov(x_2, x_n)$\\
\hline
...            & ...                       &  ... & ... \\
\hline
$cov(x_n, x_1)$ & $cov(x_n, x_2)$ & ... & $var(x_n) $\\
\hline
\end{tabular}

\end{multicols}

\pagebreak

% =============================================================================
% Page 3
% =============================================================================

\center\begin{tabular}{ |c|c|c|c|c|c| }
\hline 
Distribution & Type & Description & PMF or PDF & $E_X(x)$ & $var(x)$ \\
\hline
Bernoulli & Discrete & Binomial with n = 1 & $ p^{x}(1-p)^{1-x}$ & $p$ & $1 - p$ \\
\hline
Binomial & Discrete & n independent Bernoulli trials & $\binom{n}{x}p^{x}(1-p)^{n-x}$ & $np$ & $np(1 - p)$ \\
\hline
Geometric & Discrete & get first success in independent Bernoulli trials & $(1 - p)^{x - 1}p$ & $\frac{1}{p}$ & $\frac{1 - p}{p^2}$ \\
\hline
Negative Binomial & Discrete & get $r^{th}$ success in independent Bernoulli trials & $\binom{x - 1}{n - 1}p^{r}(1 - p)^{x - r}$ & $\frac{r}{p}$ & $\frac{r(1 - p)}{p^2}$ \\
\hline
Hypergeometric & Discrete & sampling objects w/o replacement & $\frac{\binom{A}{a}\binom{B}{b}}{\binom{N}{n}} = \frac{\binom{a}{x}\binom{N-a}{n-x}}{\binom{N}{n}}$ & $\frac{n \cdot a}{N}$ & $\frac{n \cdot a \cdot (N - a) \cdot (N - n)}{N^2 \cdot (N - 1)}$\\
\hline
Poisson & Discrete & \# of occurences of an event & $\frac{\lambda^x \cdot e^{-\lambda}}{x!}$ & $\lambda$ & $\lambda$\\
\hline
\hline
Uniform & Continuous & constant probability & $\frac{1}{b - a}, a \le x \le b$ & $\frac{a + b}{2}$ & $\frac{(b-a)^2}{12}$\\
\hline
\end{tabular}

\begin{multicols}{2}
\textbf{Maximum Likelihood Estimator}

\begin{enumerate}
\item p.d.f

Poisson p.d.f

$f(x \mid \lambda) = \frac{\lambda^x e^{-\lambda}}{x!}$

\item likelihood function 
\begin{align*}
L(\lambda) & = P(x_1 = x_1 \mid \lambda) \cdot ... \cdot P(x_n = x_n \mid \lambda) \\
           & =  \prod_{i = 1}^{n} \frac{\lambda^{x_i}e^{-\lambda}}{x_i!}
\end{align*}

\item log of likelihood
\begin{align*}
l(\lambda) & = \ln(L(\lambda)) \\
           & = \ln(\prod_{i = 1}^{n} \frac{\lambda^{x_i}e^{-\lambda}}{x_i!}) \\
           & = \sum_{i = 1}^{n} x_i \cdot ln(\lambda) - \lambda - \ln(x_i!)
\end{align*}

\item$\lambda_{MLE}$
\begin{align*}
0 & = \frac{\partial}{\partial\lambda}l(\lambda) \\
  & = \frac{\partial}{\partial\lambda} \sum_{i = 1}^{n} x_i \cdot ln(\lambda) - \lambda - \ln(x_i!) \\
  & = \sum_{i = 1}^{n} \frac{x_i}{\lambda} - 1 \\
  & = (\sum_{i = 1}^{n} \frac{x_i}{\lambda}) - n \\
  & = \frac{1}{\lambda}(\sum_{i = 1}^{n} x_i) - n \\
n & = \frac{1}{\lambda}(\sum_{i = 1}^{n} x_i) \\
\lambda_{MLE} & = \frac{\sum_{i = 1}^{n} x_i}{n}
\end{align*}

\end{enumerate}
\columnbreak

\textbf{Z-test, T-test}

Both tests measure the difference between an observed statistic and its hypothesized population parameter. 
\center\begin{tabular}{ |c|c| }
\hline 
Z-value & T-value \\
\hline
measures in units of S.D. & measures in units of S.E. \\
\hline
knows population $\sigma$ & only knows sample s\\
\hline

\hline
\end{tabular}

single value \large$z = \frac{x - \mu}{\sigma}$ \\

\begin{multicols}{2}
\small{mean of n obs} \large$z = \frac{\bar{x} - \mu}{\frac{\sigma}{\sqrt{n}}}$ \\
\small$(\bar{x} - z_{\alpha/2}\cdot\frac{\sigma}{\sqrt{n}}, \bar{x} + z_{\alpha/2}\cdot\frac{\sigma}{\sqrt{n}})$\\
\columnbreak
\large$t = \frac{\bar{x} - \mu}{\frac{s}{\sqrt{n}}}$\\

\small$(\bar{x} - t_{\alpha/2}\cdot\frac{\sigma}{\sqrt{n}}, \bar{x} + t_{\alpha/2}\cdot\frac{\sigma}{\sqrt{n}})$
\end{multicols}
\end{multicols}

\end{small}
\end{document}
